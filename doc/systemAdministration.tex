%%%% SEITENRAENDER, SCHRIFTGROESSE UND ZEILENABSTAND NICHT ABAENDERN => SONST GIBT ES PUNKTEABZUG
\documentclass[a4paper,11pt,singlespacing]{article}
% \usepackage[left=2.5cm,right=2.5cm,top=2.5cm]{geometry}
\usepackage{setspace}
\usepackage[utf8]{inputenc}
\usepackage[T1]{fontenc}
\usepackage{graphicx}
\usepackage[ngerman]{babel}
\usepackage{color}
\usepackage{wrapfig}
\usepackage{titleref}
\usepackage{hyperref}
\usepackage[rightcaption]{sidecap}
\usepackage{listings,xcolor}
\usepackage[numbers,round]{natbib}

\sloppy
\setlength{\parindent}{0ex} % Absatzeinrückung verhindern
\graphicspath{ {images/} }

\begin{document}
\pagenumbering{roman}

% Cover
\title{Systemadministration - Mailserver Honypot zum analysieren von Spam}
\author{Manuel Adams 27470, Michael Ruf 27428, Mario Waizenegger 29608}
\maketitle
\begin{abstract}
Heutzutage wird viel Spam verschickt. Einiges davon wird erkennt und gefiltert.
Das Projekt befasst sich mit der Aufgabe Spam zu erhalten und als Verteiler zu dienen, um das Verhalten der Spammer und die konkreten Spam Nachrichten zu analysieren.
\\\\
Zur Durchführung wird hierzu ein Spam Honeypot aufgesetzt.
Um nötigen Daten zu erhalten wird der Mail-Server offen publiziert und sich mit mehreren Mail-Adressen bei ominösen Diensten registriert.
\\\\
Das Ziel des Projektes ist ...\color{red}{TODO}
\end{abstract}

\newpage

% Table of contents
\tableofcontents

\newpage
\pagenumbering{arabic}

% Content
\section{Einleitung/Motivation}\label{sec:Einleitung}

	\subsection{Ziel der Arbeit}\label{sec:Ziel}
		TODO

	% TODO Nehmen wir das?!
	\subsection{Motivation}\label{sec:Motivation}
		\begin{itemize}
		\item Woher kommt Spam?
		\item Was versucht man mit Spam zu erreichen?
		\item Wer verschickt Spam?
		\item Welche Zielgruppe hat Spam?
		\item Findet man Informationen über Systeme, von denen Spam meist geschickt wird?
		\item Werden die Nachrichten generiert?
		\end{itemize}
		Diese Fragen interessieren uns.
	
	\subsection{Vorgehensweise}\label{sec:Vorgehensweise}
		TODO

	\subsection{Aufbau der Arbeit}\label{sec:Aufbau}
		TODO


\section{Grundbegriffe}\label{sec:Grundbegriffe}
	Die folgenden Begriffsdefinitionen und Unterscheidungen sind zum einen zur Verdeutlichung, wie Begriffe in dieser Arbeit verstanden werden und um Fachbegriffe zu erklären.
	
	\begin{description}
	\item[Begriffname\label{itm:BegriffReferenzname}]\hfill \\
		Beschreibung/definition des Begriffs \cite{BeispielBuch}
	\end{description}


\section{Problemstellung}\label{sec:Problemstellung}
	TODO


\section{Anforderungsanalyse/Priorisierung}\label{sec:AnforderungsanalysePriorisierung}
	TODO


\section{Lösungsvorschläge}\label{sec:Lösungsvorschläge}
	TODO


\section{Auswahl Lösung anhand Anforderungen}\label{sec:AuswahlLösungAnhandAnforderungen}
	TODO


\section{Umsetzung}\label{sec:Umsetzung}
	TODO


\section{Fazit/Ausblick/Übertragbarkeit}\label{sec:Fazit/Ausblick/Übertragbarkeit}
	TODO


% Quotes
\bibliography{zitate}
\bibliographystyle{plain}
\addcontentsline{toc}{section}{Literatur}

% Image listing
\listoffigures
\addcontentsline{toc}{section}{Abbildungsverzeichnis}

% TODO What are listings?!
\lstlistoflistings
\addcontentsline{toc}{section}{Listings}

\newpage

% Additional stuff
\section*{Anhang}\label{Anhang}
\addcontentsline{toc}{section}{Anhang}

\newpage

% Plagiarism declaration
\section*{Eidesstattliche Erklärung}\label{sec:Eidesstattliche Erklärung}


\end{document}