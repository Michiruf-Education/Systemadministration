%%%% SEITENRAENDER, SCHRIFTGROESSE UND ZEILENABSTAND NICHT ABAENDERN => SONST GIBT ES PUNKTEABZUG
\documentclass[a4paper,11pt,singlespacing]{article}
% \usepackage[left=2.5cm,right=2.5cm,top=2.5cm]{geometry}
\usepackage{setspace}
\usepackage[utf8]{inputenc}
\usepackage[T1]{fontenc}
\usepackage{graphicx}
\usepackage[ngerman]{babel}
\usepackage{color}
\usepackage{wrapfig}
\usepackage{titleref}
\usepackage{hyperref}
\usepackage[rightcaption]{sidecap}
\usepackage{listings,xcolor}
\usepackage[numbers,round]{natbib}

\sloppy
\setlength{\parindent}{0ex} % Absatzeinrückung verhindern
\graphicspath{ {images/} }

\begin{document}
\pagenumbering{roman}

% Cover
\title{Systemadministration - Mailserver Honypot zum analysieren von Spam}
\author{Manuel Adamns 27470, Michael Ruf 27428, Mario Waizenegger 29608}
\maketitle
\begin{abstract}
Dies ist die Zusammenfassung TODO
\end{abstract}

\newpage

% Table of contents
\tableofcontents

\newpage
\pagenumbering{arabic}

% Content
\section{Einleitung/Motivation}\label{sec:EinleitungMotivation}

	\subsection{Ziel der Arbeit}\label{Ziel}
		TODO
	
	\subsection{Vorgehensweise}\label{sec:Vorgehensweise}
		TODO

	\subsection{Aufbau der Arbeit}\label{sec:Aufbau}
		TODO


\section{Grundbegriffe}\label{sec:Grundbegriffe}
	Die folgenden Begriffsdefinitionen und Unterscheidungen sind zum einen zur Verdeutlichung, wie Begriffe in dieser Arbeit verstanden werden und um Fachbegriffe zu erklären.
	
	\begin{description}
	
	\item[Begriffname\label{itm:BegriffReferenzname}]\hfill \\
	Beschreibung/definition des Begriffs \cite{BeispielBuch}
	
	\end{description}


\section{Problemstellung}\label{sec:Problemstellung}
	TODO


\section{Anforderungsanalyse/Priorisierung}\label{sec:AnforderungsanalysePriorisierung}
	TODO


\section{Lösungsvorschläge}\label{sec:Lösungsvorschläge}
	TODO


\section{Auswahl Lösung anhand Anforderungen}\label{sec:AuswahlLösungAnhandAnforderungen}
	TODO


\section{Umsetzung}\label{sec:Umsetzung}
	TODO


\section{Fazit/Ausblick/Übertragbarkeit}\label{sec:Fazit/Ausblick/Übertragbarkeit}
	TODO


% Quotes
\bibliography{zitate}
\bibliographystyle{plain}
\addcontentsline{toc}{section}{Literatur}

% Image listing
\listoffigures
\addcontentsline{toc}{section}{Abbildungsverzeichnis}

% TODO What are listings?!
\lstlistoflistings
\addcontentsline{toc}{section}{Listings}

\newpage

% Additional stuff
\section*{Anhang}\label{Anhang}
\addcontentsline{toc}{section}{Anhang}

\newpage

% Plagiarism declaration
\section*{Eidesstattliche Erklärung}\label{sec:Eidesstattliche Erklärung}


\end{document}