%%%% SEITENRAENDER, SCHRIFTGROESSE UND ZEILENABSTAND NICHT ABAENDERN => SONST GIBT ES PUNKTEABZUG
\documentclass[a4paper,11pt,singlespacing]{article}
% \usepackage[left=2.5cm,right=2.5cm,top=2.5cm]{geometry}

\usepackage{setspace}
\usepackage[utf8]{inputenc}
\usepackage[T1]{fontenc}
\usepackage{graphicx}
\usepackage[ngerman]{babel}
\usepackage{color}
\usepackage{wrapfig}
\usepackage{titleref}
\usepackage{hyperref}
\usepackage[rightcaption]{sidecap}
\usepackage{listings,xcolor}
\usepackage[numbers,round]{natbib}

\graphicspath{ {images/} }

%opening	
\title{Systemadministration - Mailserver Honypot zum analysieren von Spam}
\author{Manuel Adamns 27470, Michael Ruf 27428, Mario Waizenegger 29608}

\pagenumbering{roman}
\sloppy

\begin{document}
% Absatzeinrückung verhindern
\setlength{\parindent}{0ex}


\maketitle

%%%% ZUSAMMENFASSUNG kommt zwischen \begin{abstract}  und   \end{abstract}
\begin{abstract}

\end{abstract}


\newpage
\tableofcontents
\newpage

\pagenumbering{arabic}

\section{Einleitung/Motivation}\label{sec:EinleitungMotivation}

	\subsection{Ziel der Arbeit}\label{Ziel}
	
	\subsection{Vorgehensweise}\label{sec:Vorgehensweise}

	\subsection{Aufbau der Arbeit}\label{sec:Aufbau}
	
	


\section{Grundbegriffe}\label{sec:Grundbegriffe}
	Die folgenden Begriffsdefinitionen und Unterscheidungen sind zum einen zur Verdeutlichung, wie Begriffe in dieser Arbeit verstanden werden und um Fachbegriffe zu erklären.
	
	\begin{description}
	
	\item[Begriffname\label{itm:BegriffReferenzname}]\hfill \\
	Beschreibung/definition des Begriffs \cite{BeispielBuch}
	
	\end{description}
	
	

\section{Problemstellung}\label{sec:Problemstellung}



\section{Anforderungsanalyse/Priorisierung}\label{sec:AnforderungsanalysePriorisierung}



\section{Lösungsvorschläge}\label{sec:Lösungsvorschläge}



\section{Auswahl Lösung anhand Anforderungen}\label{sec:AuswahlLösungAnhandAnforderungen}



\section{Umsetzung}\label{sec:Umsetzung}



\section{Fazit/Ausblick/Übertragbarkeit}\label{sec:Fazit/Ausblick/Übertragbarkeit}



\newpage
\bibliography{zitate}
\bibliographystyle{plain}
\addcontentsline{toc}{section}{Literatur}

\listoffigures
\addcontentsline{toc}{section}{Abbildungsverzeichnis}

\lstlistoflistings
\addcontentsline{toc}{section}{Listings}


\newpage
\section*{Anhang}\label{Anhang}
\addcontentsline{toc}{section}{Anhang}



\newpage
\section*{Eidesstattliche Erklärung}\label{sec:Eidesstattliche Erklärung}



\end{document}